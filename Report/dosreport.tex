% This is samplepaper.tex, a sample chapter demonstrating the
% LLNCS macro package for Springer Computer Science proceedings;
% Version 2.20 of 2017/10/04
%
\documentclass[runningheads]{llncs}
%
\usepackage{graphicx}
% Used for displaying a sample figure. If possible, figure files should
% be included in EPS format.
%
% If you use the hyperref package, please uncomment the following line
% to display URLs in blue roman font according to Springer's eBook style:
% \renewcommand\UrlFont{\color{blue}\rmfamily}

\begin{document}

\title{Casper CTF: Solutions}


\author{Siebe Dreesen}
\institute{}

\maketitle       

\section{Overview }
\begin{table}[]
\centering

\begin{tabular}{cll}
\multicolumn{1}{c|}{Level} & Password                                              & Time spent \\ \hline
\multicolumn{1}{c|}{3}     & \multicolumn{1}{l|}{mlHvsSM2qDSqIK1bOyixamw7O7igKdBF} & 10 minutes \\
\multicolumn{1}{c|}{4}     & \multicolumn{1}{l|}{HikOeJRoG2kYCI1QrHBf0bouflFWStPy} & 4 hours    \\
\multicolumn{1}{c|}{5}     & \multicolumn{1}{l|}{XRTOmu0ToSDrWsPMv7UFPqRcDJFUVf6E} & 8 hours    \\
\multicolumn{1}{c|}{6}     & \multicolumn{1}{l|}{}                                 &            \\
\multicolumn{1}{c|}{61}    & \multicolumn{1}{l|}{}                                 &            \\
\multicolumn{1}{c|}{62}    & \multicolumn{1}{l|}{}                                 &            \\
\multicolumn{1}{c|}{63}    & \multicolumn{1}{l|}{}                                 &            \\
\multicolumn{1}{c|}{7}     & \multicolumn{1}{l|}{}                                 &            \\
\multicolumn{1}{c|}{8}     & \multicolumn{1}{l|}{}                                 &            \\ \hline
Total                      &                                                       &           
\end{tabular}
\end{table}

\section{Casper 4 solution}
\subsection{Description}
This program takes 1 argument as input, if more are given it will print the first argument and stop the program immediately. The program contains a struct which contains a buffer of 775 characters and a function pointer. When less then 2 arguments are given the first will be copied into the buffer of the struct using strcpy() and the function pointer will be called. Normally this function pointer will point to the greetUser function which will print string.

\subsection{Vulnerability}
This program contains a memory management vulnerability, specifically a buffer overflow vulnerability. The struct contains a buffer and a function pointer which will be allocated under the buffer and the strcpy() which will copy the first argument does not do bounds checking. This means the buffer can be overflowed and the function pointer can be overwritten by giving an argument that is bigger than the allocated space for the buffer (775 bytes). 

\subsection{Exploit description}
The vulnerability can be exploited by giving a well-chosen argument that is bigger than the buffer space and overwrites the function pointer. The argument contains shell code which will spawn the /bin/xh shell. The specific argument is build from the following bytes:
\begin{itemize}
	\item A NOP sled of 755 byte
	
	\item The shell code of 22 bytes
	\item Address of the sled which will be pointed to the shell code and overwrites the function pointer
\end{itemize}
This argument will be copied to the buffer and overwrite the function pointer with the right address. By using a NOP sled the exploit becomes more robust because any address in the sled can be called to execute the shell code.

\subsection{Mitigation}
A possible mitigation would be to use strncpy() instead of srtcpy() which will only copy N characters and the buffer will never overflow. Other possible mitigations are non-executable data so that the shell code cannot be executed or address space layout randomisation which will make it harder to point to the right address of the shellcode.

\section{Casper 5 solution}
\subsection{Description}
This program again takes 1 input which is expected to be of a user. The program copies this name to the dedicated buffer using the unsafe strcpy(). The program stores users in a struct containing a name and a pointer to a role which contains the name of the role and the authority level. If the user has authority level 1 the /bin/xh shell will run. The default role has authority 0.

\subsection{Vulnerability}
The vulnerability is again because of the unsafe strcpy() which let's the attacker possibly overwrite the role pointer to a role struct with authority 1. This will cause the shell to automatically execute.

\subsection{Exploit description}
I have chose for a data-only attack by corrupting the role pointer. I did this by giving an argument which had 776 chars and 1 address. The cars overflowed the allocated space for name causing the role pointer to be overwritten by the chosen address.  

This problem took longer than expected because I first tried to exploit this by letting this point to a custom role struct in the name part and letting the pointer point to that struct instead of the default. This did not work because you cannot give an argument with null bytes which are necessary to represent integer 1.

So the solution was to find a couple of bytes of the right length in the memory that already contained int 1. I did this by searching in the c library. I was able to find an address which had 32 random bytes and then 4 bytes which represented 1. I gave this address as argument and the shell executed.

\subsection{Mitigation}
The easiest defence would again be to not use strcpy() but strncpy() instead, this would make sure the attacker cannot overflow the name by a given argument. An other mitigation would be to use address space layout randomisation which would make it very hard for the attacker to find the address of certain bytes that would have the role structure.

\section{Casper 6 solution}
\subsection{Description}
\subsection{Vulnerability}
\subsection{Exploit description}
\subsection{Mitigation}
\subsection{Advanced levels}

\section{Casper 7 solution}
\subsection{Description}
\subsection{Vulnerability}
\subsection{Exploit description}
\subsection{Mitigation}

\section{Casper 8 solution}
\subsection{Description}
\subsection{Vulnerability}
\subsection{Exploit description}
\subsection{Mitigation}

\end{document}
